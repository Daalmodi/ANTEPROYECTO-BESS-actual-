Para dar cumplimiento a los objetivos planteados en el proyecto , se establecen  las siguientes fases  (Figura 7):

\begin{figure}[h!]
    \begin{center}
    \centering
    \includegraphics[scale=0.7]{Imágenes/Metodología/Fases proyecto BESS.eps}
	\caption{Fases proyecto BESS}
    \end{center}
\end{figure}



\begin{itemize}
    \item 
    \textbf{Fase 1 :} Se indaga acerca del comportamiento  de sistemas de almacenamiento por baterías bajo la influencia de la red eléctrica 
    para realizar una comparación y selección de tecnologías de baterías existentes en el mercado, teniendo en cuenta factores tales como la eficiencia, densidad de energía y ciclo de vida.
    \item 
   \textbf{Fase 2 : }Se establecen escenarios de prueba a través de diferentes topologías simuladas de sistemas de almacenamiento por baterías considerando los fenómenos de calidad de potencia de la red eléctrica que se presentan en la zona de estudio. Posteriormente , se procede a realizar cambios en los escenarios de prueba mediante modificaciones en las topologías con la finalidad de evaluar el comportamiento y desempeño de la batería frente a una alteración en la red eléctrica.
\newpage    
    \item 
    \textbf{Fase 3 : } Se realiza una  elección de la topología más adecuada del sistema de almacenamiento por baterías teniendo en cuenta las tecnologías empleadas en los diferentes sistemas de conversión de potencia, fenomenos de calidad de potencia en la zona , especificaciones técnicas de las baterías , costos  y condiciones medioambientales, entre otros. Posteriormente , simular la topología del sistema mediante un software para verificar su óptimo funcionamiento.
\end{itemize}

